\documentclass[11pt,a4paper]{article}

\usepackage{style2017}
\usepackage{hyperref}

\hypersetup{
    colorlinks =false,
    linkcolor=blue,
   linkbordercolor = 1 0 0
}
\newcounter{numexo}
\setcellgapes{1pt}

\begin{document}



\begin{NSI}
{Exercice}{Réseau et Internet}
\end{NSI}
\addtocounter{numexo}{1}
\subsection*{\Large Exercice \thenumexo \hspace{1cm}\textit{Sur feuille}}
On souhaite raccorder 1000 machines dans un même sous réseau IP. 
\begin{enumerate}
\item Donner le plus petit masque permettant de définir un tel sous réseau.
\item Une machine A de ce réseau a pour adresse IP : 192.168.1.25. Déterminer l'adresse réseau de ce sous réseau.
\item Une machine B a pour adresse IP : 192.168.3.220. Fait-elle partie du même sous réseau que la machine A ?
\end{enumerate}

\addtocounter{numexo}{1}
\subsection*{\Large Exercice \thenumexo \hspace{1cm}\textit{Sur feuille}}
On a 5 machines A, B, C, D et E qui ont pour adresses respectives :\medskip

\begin{tabular}{p{6cm}p{6cm}}
\textbf{A~}129.186.123.9 & \textbf{B}~129.186.128.7 \\
\textbf{C}~129.186.126.8 & \textbf{D}~129.186.125.4 \\
\textbf{E}~129.186.129.1
\end{tabular}\medskip

Le masque de sous réseau de ces machines est 255.255.248.0. Répartissez les machines par sous réseau.
\bigskip
\bigskip



\noindent\fbox{
\begin{minipage}{1\textwidth}\smallskip
Le logiciel \textbf{filius} permet de créer virtuellement des réseaux avec des machines, serveurs, commutateurs et routeurs. Il est également possible de faire tourner des services (web, ftp,etc). Il est assez simple à prendre en main et s'utilise en deux temps :
\begin{enumerate}
\item En \textbf{mode construction} (clic sur le marteau) pour configurer les machines et le réseau;
\item En \textbf{mode simulation} (clic sur la flèche verte) pour voir le fonctionnement d'un réseau.
\end{enumerate}
Ce logiciel permet aussi une analyse des trames sur le réseau.\smallskip

Consultez la documentation se trouvant sur le site \href{https://github.com/KELLERStephane/KELLER-Stephane-Tests2maths/blob/master/6\%20-\%20Filius/Filius\%20guide\%20du\%20debutant.pdf}{\blue{Documentation filius}} pour vous aider.\smallskip
\end{minipage}
}


%\addtocounter{numexo}{1}
%\subsection*{\Large Exercice \thenumexo }
%\begin{enumerate}
%\item Créer un réseau nommé JULES contenant trois PC dont l'adresse réseau est 192.168.0.0/24. Les adresses IP des trois PC ont un dernier octet compris entre 10 et 20.
%\item Au réseau JULES précédent, ajouter un serveur d'adresse IP : 192.168.0.1
%\item Le serveur est-il accessible depuis les trois machines ? Vérifiez avec un ping.
%\item Sur le serveur, ajoutez l'application "serveur générique" et démarrez le service.
%\item Sur la machine A, ajoutez l'application "client générique" en configurant l'adresse IP du serveur puis se connecter.  Envoyer un message et vérifiez la bonne réception sur le serveur.
%\item Faites un clic droit sur le serveur et afficher les échanges de données. 
%\item Envoyez un nouveau message et observez la table d'échanges de données. 
%
%En sélectionnant une ligne de la table, le descriptif des différentes trames apparait.
%
%Examinez chacune d'elle attentivement, repérez les 4 couches et les différents protocoles.
%\end{enumerate}


\addtocounter{numexo}{1}
\subsection*{\Large Exercice \thenumexo }
\begin{enumerate}
\item Créer un réseau nommé JULES contenant trois PC dont l'adresse réseau est 192.168.0.0/24. Les adresses IP des trois PC ont un dernier octet compris entre 10 et 20.
\item Au réseau JULES, ajouter un serveur d'adresse IP : 192.168.0.1

Vérifier que les PC communiquent avec le serveur.
\item Ajouter sur le serveur l'application "Serveur web" et démarrer le service.
\item Ajouter sur la machine A l'application "Navigateur web". 

Lancez le navigateur et saisissez l'adresse IP du serveur web pour afficher la page web.
\item Afficher la table d'échange des données. Pensez à vider les tables de temps à autre.

\textbf{Remarque :} Vous pouvez réduire la vitesse de transmission des trames en diminuant le pourcentage (barre horizontale située à droite du bouton simulation).
\end{enumerate}



\addtocounter{numexo}{1}
\subsection*{\Large Exercice \thenumexo }
L'accès à un serveur web par son adresse IP est possible mais en général on utilise une url écrite de façon à pouvoir la mémoriser facilement. La transformation de l'url en adresse IP se fait avec un serveur DNS.
\begin{enumerate}
\item Ajouter un second serveur au réseau nommé "Serveur DNS" dont l'adresse IP est : 192.168.0.2 avec le même masque de sous réseau.
\item Repasser en mode simulation et ajouter l'application "Serveur DNS" au serveur dédié.
\item Lancez l'application "Serveur DNS" et élargissez la fenêtre pour bien voir les 3 onglets.
\item Dans le premier onglet, ajouter le nom de domaine "www.serveur-nsi.fr," l'adresse IP du serveur web puis cliquer sur ajouter. Démarrer le service.
\item Sur la machine A, vérifiez que la page web s'affiche en saisissant http://www.serveur-nsi.fr .
\item Si vous avez rencontré un problème :
\begin{enumerate}
\item vérifiez que le service DNS est bien démarré;
\item que le nom de domaine est correctement écrit;
\item que l'adresse IP du serveur DNS est bien renseigné sur la machine A (et les autres aussi).
\end{enumerate}
\item Affichez la table d'échange de données de la machine A et des deux serveurs. Quel est le protocole utilisé par le serveur DNS ?
\end{enumerate}

\addtocounter{numexo}{1}
\subsection*{\Large Exercice \thenumexo }
\begin{enumerate}
\item Créer un second réseau DUMONT composé de 2 portables, un serveur web et un commutateur selon les paramètres suivants:
\begin{enumerate}
\item adresse de réseau : 172.16.1.0
\item masque de sous reseau : 255.255.255.0
\item passerelle : 172.16.0.1
\item serveur web : 172.16.0.5
\end{enumerate}
\item Ajouter un routeur pour relier les deux réseaux. Celui-ci aura 2 interfaces réseaux.
\begin{enumerate}
\item une interface vers le réseau JULES d'ip 192.168.0.254
\item une interface vers le réseau DUMONT d'ip 172.16.0.254
\end{enumerate}
\item Le ping du portable de Dumont vers la machine A de Jules est-il un succès ? Pourquoi ?
\item Pour communiquer entre 2 réseaux, il faut que le routeur soit identifié sur chaque machine du réseau. 

Renseigner sur chaque machine des réseaux JULES et DUMONT la passerelle avec les adresses IP du routeur.
\item Vérifier qu'un ping du portable vers la machine A est fonctionnel.
\item Ajouter l'application "Navigateur web" sur le portable est afficher la page web http://www.serveur-nsi.fr. En cas d'erreur, vérifier que tout est bien renseigné sur le portable.
\end{enumerate}

%\addtocounter{numexo}{1}
%\subsection*{\Large Exercice \thenumexo }
%\begin{enumerate}
%\item Sur le Rpi, récupérer sur l'ENT l'archive "socket.zip" et la décompresser dans le dossier Documents de votre home.
%\item Ouvrir le fichier "serveur.py" sur le Rpi et compléter la variable MON\_IP avec l'adresse IP du Rpi.
%\item Sur une machine windows de la classe, récupérer sur l'ENT l'archive "socket.zip" et la décompresser dans le dossier Documents ou le dossier tmp (à la racine de c).
%\item Ouvrir le fichier "client.py" sur la machine windows et compléter la variable SERVEUR avec l'adresse IP du Rpi.
%\item Exécuter les programmes pythons serveur.py sur le Rpi et client.py sur la machine windows. 
%\item Les deux machines sont connectées et peuvent communiquer. Vous pouvez saisir des messages entourés par les guillemets !
%\item Arrêter les programmes puis saisir l'adresse IP du Rpi d'un autre binôme pour communiquer.
%\end{enumerate}
\end{document}

